\chapter{Caffe}

\section*{Introduction}
	本章节内容主要是关于caffe框架的一些知识。
	

\section{lmdb}
	lmdb格式的文件在使用Python程序进行生成的时候,如果需要重复生成,则需要先删除原来的。否则会在原先的lmdb上重新添加文件。


	
\section{net protobuf}
	\boldmath  %公式加粗
	此部分内容有待添加
	
	

\section{solver}	
	solver文件为caffe的训练参数的文件,主要存储一些训练的超参数
    
    运行代码为:# caffe train --solver=*slover.prototxt
    
    一个例子:
    
    \begin{lstlisting}
    train_net: "lenet_train.prototxt"
    test_net: "lenet_test.prototxt"	
	test_iter: 100
	test_interval: 500
	
	base_lr: 0.01
	lr_policy: "fixed"
	
	momentum: 0.9
	type: SGD
	weight_decay: 0.0005
	display: 100
	max_iter: 20000
	snapshot: 5000
	snapshot_prefix: "models"
	solver_mode: CPU
	\end{lstlisting}
	
	下面来一个一个解释这些程序的意思
    
    1.    
        
    train\_ net: "lenet\_ train.prototxt"
    
    test\_ net: "lenet\_ test.prototxt"
    	
    这两行用于定于训练网络和测试网络,可以是同一个网络,用net:train\_ test.prototxt 来表示,为上一节的内容。注意:文件的路径要从caffe的根目录开始,其他所有的配置都是这样的。
    
    2.接下来test\_ iter 表示一次训练需要加载多少个数据,许训练中的batch size是一致的;test\_ iterval表示经过多少此训练的Iteration后进行一次测试,如500表示没经过500个Iteration进行一次测试。另外,如果网络不想进行测试的话,可以在solver文件中加入如下的参数 test\_ initialization: false 这样不管经过多少次的Iteration都不会进行测试。
    
    3.有关于learing rate的东西
    
    base\_ lr 表示初始的一个learing rate,如果lr\_ policy 如果设置为fixed,训练过程中会一直维持这个learing rate不再改变,其他的都是会在训练过程中逐渐变化的。
    
    lr\_ policy 可设置的值如下所示:
    \begin{itemize}
    	\item fixed:保持base\_ lr不变
    	\item step: 如果设置为step,则还需要设置一个stepsize,返回 $base\_ lr*gamma^{floor*         \frac{iter}{stepsize}}$,其中iter表示当前的迭代次数。如设置gamma=0.9,stepsize=100表示在第一百次迭代后learning rate 下降到原来的0.9倍
    	\item exp:返回$base\_ lr * gamma ^ iter$, iter为当前迭代次数
    	\item inv:如果设置为inv,还需要设置一个power, 返回$base_lr * (1 + gamma * iter) ^ {- power}$
    	\item multistep:如果设置为multistep,则还需要设置一个stepvalue。这个参数和step很相似,step是均匀等间隔变化,而multistep则是根据stepvalue值变化
    	\item poly:学习率进行多项式误差, 返回 $base\_lr (1 - iter/max\_ iter) ^ {power}$
    	\item sigmoid:学习率进行sigmod衰减,返回 $base\_lr ( 1/(1 + exp(-gamma * (iter - stepsize))))$
    
    \end{itemize}
    需要设置参数的数量随着lr pilicy的不同而有所变化。如设置为fixed则不需要添加任何参数,设置为step则需要添加gamma和stepsize两个参数,设置为step的策略后solver配置如下所示:
    \begin{lstlisting}
	base_lr: 0.01
	lr_policy: "step"
	gamma:0.9
	stepsize:100
	\end{lstlisting}
	
	4.对于momentum,一般取值在0.5--0.99之间。通常设为0.9,momentum可以让使用SGD的深度学习方法更加稳定以及快速。详细的资料,参考Hinton的论文《A Practical Guide to Training Restricted Boltzmann Machines》
    
    5. type:SGD
    
    表示优化算法,总共有六种:SGD、AdaDelta、AdaGrad、Adam、NAG、RMSprop
    
    6.weight\_ decay 为权重衰减项,详细的内容已经在上一章解释过了。
    
    7.display:100表示没训练100个Iteration显示一次loss
    
    8.max\_ iter:2000 表示最大的迭代此时为2000
    
    9.
    
    snapshot:500
    
    snapshot\_ prefix :"models" 
    
    表示没训练500个Iteration,保存一次网络的参数数据,保存路径为models。同时会保存另外一个solverstate文件,以便下次训练的时候可以从这一步继续训练。
    
    10.solver\_ mode: CPU设置运行模式为CPU




\section{一些其他的问题}
	\subsection{loss=87.3365}
		\begin{itemize}
			\item 如果一开始就出现这种情况,有可能是参数的初始化方式不对,使用xavier可以避免这种情况
			\item 在caffe中数据的标签必须从0开始,且为整数(1.0,2.0也可以)
			\item 数据可能会出现问题(一定要反复确定数据没有问题才可以)
			\item 如果是图片数据可能是图片没有进行归一化,最好是进行归一化并且减去均值
			\item 如果一开始的loss是在下降的,但是训练到后期出现了87.3365这个值,代表网络发散了,需要调整learing rate到一个较小的值,建议从0.01开始,到0.0001即可适合大部分网络结构
		\end{itemize}
		
	\subsection{一些差别}
		solver.step(1) 表示网络进行一次迭代,即进行一次前向过程和一次反向传播
		
		solver.net.forward() 表示网络只进行一次前向的过程,可用于网络的预测
		
	\subsection{多GPU计算}
		如果使用solver文件:build/tools/caffe train --solver=models/bvlc\_ alexnet/solver.prototxt --gpu=0,1
		
		如果使用Python文件:python yourpythonfile.py CUDA\_ VISIBLE\_ DEVICES=0, 1
	




	% \begin{enumerate}		
	% 	\item start with weights $w_i = \frac{1}{N} \quad i=1...N$
	% 	\item repeat for m=1 to M
	% 		\begin{itemize}
	% 			\item 使用输入数据训练一个分类器$G_m(x) \in (-1,1)$
	% 			\item 计算误差err:
	% 				\begin{equation*}
	% 					E_w[I_(y_i\not \equiv G_m(x_i))]=\frac{\sum_{i=1}^{N}w_i I_(y_i\not \equiv G_m(x_i))}{\sum_{i=1}^{N}w_i}
	% 				\end{equation*}
	% 			\item 输出$\alpha_m = \frac{1}{2}log(\frac{1-err}{err})$,从这里可以看出分类器的误差越大,权值越小
	% 			\item set $w_i = \frac{w_i}{Z_m} \exp(\alpha_m I_(y\not \equiv G_m(x))$,其中$Z_m$是规范化因子,\newline
	% 			$Z_m=\sum_{i=1}^{N}w_i \exp(-\alpha_m y_i G_m(x_i))$如果一个数据点分类错了,那么给这个点的权值大一点。
	% 		\end{itemize}
	% 	\item 这样我们就得到了$G(x)=sign[\sum_{m=1}^{M}\alpha_m G_m(x)]$
	% \end{enumerate}



%--------------------------------------------------------------------------------
%--------------------------------------------------------------------------------
%--------------------------------------------------------------------------------