\chapter{预处理以及一些注意事项}

\section*{Introduction}
	本章节内容主要介绍一些机器学习方法中常用的预处理方法以及一些注意事项

\section{预处理}
	\subsection{特征提取方法}
	常用的特征提取方法如下:
	
	\begin{itemize}
		\item 基于互信息(也叫做皮尔逊相关系数,信息增益)提取特征:$I(X,Y) = E(Y) - E(Y|X)$
		\item 基于决策树进行目标特征的选择(GBDT)
		\item L1,利用L1离散化特征,再使用L2进行交叉验证
		\item RF和LR对模型特征进行打分
		\item 基于深度学习,进行自动选择特征
		\item 特征是否发散:方差接近于0,不采用
	\end{itemize}
	
	\subsection{特征标准化}
	常用的特征标准化方法如下:
	
	\begin{itemize}
		\item 对于逻辑回归或者神经网络来说,需要将数据归一化到[-1,1]
		\item 对贝叶斯来说需要将数据归一化到(0,1),表达是或者不是
		\item 对于连续的特征来说,归一化可使用如下的公式:$x^* = \frac{x-\mu}{s}$,其中如果归一化到(0,1)的话,s为方差;如果归一化到[-1,1],$s = x_max - x_min$
		\item 对于非连续化的特征来说,采用one-hot编码实现
	\end{itemize}