\chapter{Deep Learing}

\section*{Introduction}
	本章节内容主要是来自于深度学习中遇到的一些坑以及问题
	
\section{一些小的trick}

1. 一定要对数据进行归一化

2. 训练可能会出现loss一开始迅速下降,然后稳定在一定的值上,这时候不要以为网络已经收敛了,有时候会出现一个拐点,即在训练很后面的时候会重新出现一个新的下降的区间,这时候网络基本上会收敛。但是不排除会出现多的拐点的情况。结论:训练过程中一定要耐心耐心再耐心。
	
\section{weight decay}
	\boldmath  %公式加粗
	在损失函数中,weight decay是放在正则项(regularization)前面的一个系数,正则项一般指示模型的复杂度,所以weight decay的作用是调节模型复杂度对损失函数的影响,若weight decay很大,则复杂的模型损失函数的值也就大。我所理解的weight decay就是一个调节正则化项的系数,增大则对正则化项的依赖会更高,不容易过拟合,反之则反之。
	
	利用weight decay给损失函数加了个惩罚项,使得在常规损失函数值相同的情况下,学习算法更倾向于选择更简单(即权值和更小)的NN。是一种减小训练过拟合的方法。
	
	Weight decay is equivalent to L2 regularizer.
	
	在训练神经网络的时候,可以先设置weight decay为0,然后使网络过拟合,查看测试结果。结果正确后,设置weight decay为大一点的值,即加入正则化项,这样可以防止过拟合现象。具体大小需要在网络中调试。
	
	遇到的问题:在训练U-net的时候,出现训练结果全是黑图的情况。原因是weight decay设置过大,导致网络结果没有过拟合。解决方案:设置weight decay为更小的值。
	

\section{momentum}
	momentum是梯度下降法中一种常用的加速技术。对于一般的SGD,其表达式为$x \gets x-\alpha * dx$,x沿负梯度下降。而带momentum项的SGD则写生如下形式:$v=  \beta *v -a*dx$,$x \gets x+v$,其中$\beta$是momentum系数,通俗的理解上面式子就是,如果上一次的momentum(即v)与这一次的负梯度方向是相同的,那这次下降的幅度就会加大,所以这样做能够达到加速收敛的过程。
	
	神经网络的训练过程(也就是梯度下降法)是在高维曲面上寻找全局最优解的过程(也就是寻找波谷),每经过一次训练epoch,搜寻点应该更加靠近最优点所在的区域范围,这时进行权重衰减便有利于将搜寻范围限制在该范围内,而不至于跳出这个搜索圈,反复进行权重衰减便逐渐缩小搜索范围,最终找到全局最优解对应的点,网络收敛。momentum是冲量单元,也就是下式中的m,作用是有助于训练过程中逃离局部最小值,使网络能够更快速地收敛,也是需要经过反复地trial and error获得的经验值
	
	主要作用:防止陷入局部最小值








	% \begin{enumerate}		
	% 	\item start with weights $w_i = \frac{1}{N} \quad i=1...N$
	% 	\item repeat for m=1 to M
	% 		\begin{itemize}
	% 			\item 使用输入数据训练一个分类器$G_m(x) \in (-1,1)$
	% 			\item 计算误差err:
	% 				\begin{equation*}
	% 					E_w[I_(y_i\not \equiv G_m(x_i))]=\frac{\sum_{i=1}^{N}w_i I_(y_i\not \equiv G_m(x_i))}{\sum_{i=1}^{N}w_i}
	% 				\end{equation*}
	% 			\item 输出$\alpha_m = \frac{1}{2}log(\frac{1-err}{err})$,从这里可以看出分类器的误差越大,权值越小
	% 			\item set $w_i = \frac{w_i}{Z_m} \exp(\alpha_m I_(y\not \equiv G_m(x))$,其中$Z_m$是规范化因子,\newline
	% 			$Z_m=\sum_{i=1}^{N}w_i \exp(-\alpha_m y_i G_m(x_i))$如果一个数据点分类错了,那么给这个点的权值大一点。
	% 		\end{itemize}
	% 	\item 这样我们就得到了$G(x)=sign[\sum_{m=1}^{M}\alpha_m G_m(x)]$
	% \end{enumerate}



%--------------------------------------------------------------------------------
%--------------------------------------------------------------------------------
%--------------------------------------------------------------------------------